%----------------------------------------------------------------------------------------
%	翻译:
%   校对:未校对!
%----------------------------------------------------------------------------------------


\chapter{形式关系与样例系统}
\label{chap3}

\section{热力学Euler方程}
\label{sec3.1}
上一章从基本假设出发解出了平衡态条件,下面深入探究基本方程的数学特征。

基本方程的一阶齐次性使得它可以写成更方便的形式,称为Euler形式。

从一阶齐次性的定义出发,对任意常数$\lambda$都有:
\begin{equation}
\label{equ3.1}
    U(\lambda S, \lambda X_1, \dots, \lambda X_t) = \lambda U(S, X_1, \dots, X_t).
\end{equation}
等式两侧对$\lambda$求导:
\begin{align}
    \frac{ \partial U(\dots, \lambda X_k, \dots)}{\partial (\lambda S)} \frac{\partial (\lambda S)}{\partial \lambda} + \frac{ \partial U(\dots, \lambda X_k, \dots)}{\partial (\lambda X_j)} \frac{\partial (\lambda X_j)}{\partial \lambda} \notag \\
    + \dots = U(S, X_1, \dots, X_t) \label{equ3.2} \\
    \frac{\partial U(\dots, \lambda X_k, \dots)}{\partial (\lambda S)} S + \sum_{j = 1}^t \frac{\partial U(\dots, \lambda X_k, \dots)}{\partial (\lambda X_j)} X_j \notag \\
    = U(S, X_1, \dots, X_t) \label{equ3.3}
\end{align}
方程对任意$\lambda$都成立,取$\lambda = 1$可得:
\begin{align}
    \pUpS S + \sum_{j = 1}^t \frac{\partial U}{\partial X_j} X_j + \dots = U \label{equ3.4} \\
    U = TS + \sum_{j = 1}^t P_j X_j \label{equ3.5}
\end{align}
对于简单系统,上式写为
\begin{equation}
\label{equ3.6}
    U = TS - PV + \mu_1 N_1 + \dots + \mu_r N_r
\end{equation}
\eqref{equ3.5}或\eqref{equ3.6}式是齐次函数Euler定理\mpar{若函数$f(x, y, \dots)$满足$f(\lambda x, \lambda y, \dots) = \lambda^n f(x, y, \dots)$, 则$f$称为$n$阶齐次的。齐次函数Euler定理为$x \frac{\partial f}{\partial x} + y \frac{\partial f}{\partial y} + \dots = n f$.}的一阶齐次情形在热力学理论的应用。公式的推导过程即为齐次定理的证明过程。\eqref{equ3.5}或\eqref{equ3.6}式称为(热力学)Euler关系。

类似地,熵表象下Euler关系的形式为
\begin{align}
    S &= \sum_{j = 0}^t F_j X_j \label{equ3.7} \\
    S &= \left(\frac{1}{T} \right) U + \left( \frac{P}{T} \right) V - \sum_{k = 1}^r \left( \frac{\mu_k}{T} \right) N_k. \label{equ3.8}
\end{align}

\ 

{\bf \Large 习题}

\ 

\begin{itemize}
\item[3.1-1.] 写出习题1.10-1当中具有物理意义的基本方程的Euler形式。
\end{itemize}

\section{Gibbs-Duhem关系}
\label{sec3.2}
第二章导出了用温度、压强和化学势表示的平衡条件。这些强度量的引入过程比较相似,事实上,它们的形式体系也是对称的。尽管号称有对称性,但我们对温度与压强有着非常直观的感受,而对化学势就差了一点。有趣的是,这些强度量之间不是完全独立的,它们之间存在函数关系,例如单组分系统的化学势$\mu$可表示为$T, P$的函数。

这种关系是基本方程一阶齐次性的结果。考虑某一单组分系统,基本方程可写为$u = u(s, v)$(即\eqref{equ2.19}式);三个强度量也都是$s, v$的函数,原则上这三个状态方程
\begin{align*}
    T &= T(u, v) \\
    P &= P(u, v) \\
    \mu &= \mu (u, v)
\end{align*}
可消去$u, v$形成一个关于$T, P, \mu$的方程。

容易推广到一般情况,关键还是数清楚变量与方程的数目。设基本方程具有$t + 1$个广延量:
\begin{equation}
\label{equ3.9}
    U = U(S, X_1, X_2, \dots, X_t).
\end{equation}
由此产生$t + 1$个状态方程:
\begin{equation}
\label{equ3.10}
    P_k = P_k(S, X_1, X_2, \dots, X_t).
\end{equation}
令\eqref{equ2.14}式中的任意参量$\lambda$为$\lambda = 1 / X_t$,可得
\begin{equation}
\label{equ3.11}
	P_k = P_k \left( \frac{S}{X_t}, \frac{X_1}{X_t}, \dots, \frac{X_{t-1}}{X_t}, 1 \right).
\end{equation}
可见这$t + 1$个强度量都是关于$t$个变量的函数,从$t + 1$个方程中消去$t$个变量就得到强度量之间的关系。

知道基本方程的具体形式就能求出强度量之间关系的具体形式。给定基本方程之后的套路即为$\eqref{equ3.9} \sim \eqref{equ3.11}$式的过程。

这种关系的微分形式(称为{\bf Gibbs-Duhem关系})可以从Euler关系直接导出。对\eqref{equ3.5}式微分得到
\begin{equation}
\label{equ3.12}
	dU = TdS + SdT + \sum_{j = 1}^t P_j dX_j + \sum_{j = 1}^t X_j dP_j.
\end{equation}
由\eqref{equ2.6}式可得
\begin{equation}
\label{equ3.13}
	dU = TdS + \sum_{j = 1}^t P_j dX_j.
\end{equation}
以上两式相减即得到Gibbs-Duhem关系:
\begin{equation}
\label{equ3.14}
	SdT + \sum_{j = 1}^t X_j dP_j = 0.
\end{equation}
对于单组分简单系统有
\begin{equation}
\label{equ3.15}
	SdT - VdP + Nd\mu = 0.
\end{equation}
或者
\begin{equation}
\label{equ3.16}
	d\mu = -sdT + vdP
\end{equation}
可见化学势的变化与温度及压强的变化有关,而非独立变化,并且$\mu, T, P$三者中已知任何两个的变化就能确定其余一个的变化。

Gibbs-Duhem关系是强度量之间关系的微分形式,将该式积分即得到显式形式,这是从$\eqref{equ3.9} \sim \eqref{equ3.11}$式的另一种计算套路。Gibbs-Duhem关系的积分:
\[
	\int S(T, P_1, \dots, P_t) dT + \sum_{j = 1}^t \int X_j(T, P_1, \dots, P_t) dP_j = 0
\]
需要知道各广延量$X_j$用强度量$P_j$表示的形式,这可以从状态方程\mpar{状态方程:强度量=强度量(广延量)}解出。因此积分Gibbs-Duhem关系必须知道系统的状态方程。

系统可独立变化的强度量个数称为系统的{\it 热力学自由度 (thermodynamic degrees of freedom)}。{\it 一个具有$r$种组分的简单系统的热力学自由度为$r + 1$.}

熵表象下的Gibbs-Duhem关系相似:
\begin{align}
	&\sum_{j = 0}^t X_j dF_j = 0 \label{equ3.17} \\
	&U d\left( \frac{1}{T} \right) + V d\left( \frac{P}{T} \right) - \sum_{k = 1}^r N_k d\left( \frac{\mu_k}{T}\right) = 0 \label{equ3.18}
\end{align}

\ 

{\bf \Large 习题}

\ 

\begin{itemize}
\item[3.2-1.] 某系统的基本方程为
\[
	U = \left( \frac{v_0^2 \theta}{R^3} \right) \frac{S^4}{NV^2}
\]
求$T, P, \mu$之间的函数关系。
\end{itemize}

\section{形式关系总结}
\label{sec3.3}
现在总结一下能量表象的热力学体系结构。简明起见,考虑单组分的简单系统,它的基本方程
\begin{equation}
\label{equ3.19}
    U = U(S, V, N)
\end{equation}
包含了该系统所有的热力学信息。定义了温度$T \equiv \partial U / \partial S$等强度量之后,从基本方程可以导出三个状态方程:
\begin{align}
    T &= T(S, V, N) = T(s, v) \label{equ3.20} \\
    P &= P(S, V, N) = S(s, v) \label{equ3.21} \\
    \mu &= \mu(S, V, N) = \mu(s, v) \label{equ3.22}
\end{align}

\section{单组分/多组分简单理想气体}
\label{sec3.4}

\section{理想van der Waals流体}
\label{sec3.5}

\section{黑体辐射系统}
\label{sec3.6}

\section{橡胶带系统}
\label{sec3.7}

\section{不可控变量;磁系统}
\label{sec3.8}

\section{摩尔热容与其他导出量}
\label{sec3.9}