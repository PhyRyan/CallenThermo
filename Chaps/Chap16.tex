%!TEX root = ../CallenThermo.tex
\chapter{正则系综:Helmholtz 表象下的统计力学}\label{chap16}
\section{概率分布}\label{sec16.1}
上一章所讲的微正则系综在原理上看起来很简单,但实际计算起来却仅仅对少数几种高度理想化的系统适用。计算在任意大小的“盒子”里分配给定能量的方案数量往往超出了我们的数学能力。而将能量给定这个约束去除的方案是存在的——考虑与热库所接触的系统,而不是绝热的。与热库相接触的系统的统计力学可以看做是在“Helmholtz表象”下的,或者用这儿的术语,在{\it 正则系综理论~(canonical formalism)}下的。

对于一个与热库接触的系统而言,从零到任意高能量的全部态都是可能的。但是,与封闭系统相比,这里在各个态上并{\it 没}有相同的概率。即,系统在各个态上停留的时间并不相同。正则系综的关键在于确定系统处在诸微观态上的概率。这可以通过考察这个系统与热库共同组成的{\it 封闭}系统来得到解决,对于这个大系统,微观态的等概率原理依旧适用。

我们可以通过一个简单的例子来说明。考虑三个骰子,其中一个是红的,另两个白的。三个骰子都掷了数千次。记录当且仅当三个骰子数之和为$12$时,红色骰子的点数。那么红色骰子点数为一、二、\dots 、六的频率各是多少呢?

结果留给读者:红色骰子点数为一的频率是$2/25$、为二的概率为$3/25$、\dots 、为五的概率为$6/25$,为六的概率为$5/25$。在这个约束下,掷出一个六点的红骰子概率是$1/5$。

这个红色骰子就正如我们所关心的系统一样,而白色骰子好比热源,点数对应于能量,总点数为$12$的限制又与(系统与热库的)总能量为常数类似。

{\it 子系统处于宏观状$j$的概率$f_j$,等于子系统处于宏观状$j$(其能量为$E_j$)的微观状态数比上系统和热库全部微观状态数:}
\begin{equation}
f_j = \frac{\Omega_\text{热库}(E_\text{tot}-E_j)}{\Omega_\text{tot}(E_\text{tot})},
\end{equation}
这里$E_\text{tot}$为系统与热库的能量之和,$\Omega_\text{tot}$为系统与热库的全部状态数。分子上的$\Omega_\text{热库}(E_\text{tot}-E_j)$为子系统处于状态$j$时(给热库留下了$E_\text{tot}-E_j$的能量)热库的全部可能状态数。

这是正则系综理论里最精华%
\mpar{译注:原文为`seminal',意为含精液的,引申为有重大意义的。}%
的一个关系式,其可以重新改写成一个更便利的形式。分母项可以通过\eqref{equ15.1}式用复合系统的熵来表达,而分子和热库的熵相关。这样我们有
\begin{equation}
f_j = \frac{\exp\left\{k_\text{B}^{-1}S_\text{热库}(E_\text{tot}-E_j)\right\}}{\exp\left\{k_\text{B}^{-1}S_\text{tot}(E_\text{tot})\right\}} ,
\end{equation}
记$U$为子系统的平均能量,从熵的可加性,我们有
\begin{equation}
S_\text{tot}(E_\text{tot}) = S(U) + S_\text{热库}(E_\text{tot}-U).
\end{equation}
此外,将熵$S_\text{热库}(E_\text{tot}-E_j)$在平衡点$E_\text{tot}-U$附近展开
\begin{equation}
\begin{aligned}
S_\text{热库}(E_\text{tot}-E_j)& = S_\text{热库}(E_\text{tot}-U+U-E_j)\\
&= S_\text{热库}(E_\text{tot}-U) + (U-E_j)/T
\end{aligned}
\end{equation}
展开式中不存在更多的项(这也是热库的定义)。将后两个方程带入$f_j$的表达式
\begin{equation}
f_j = {\mathrm e}^{\{U-TS(U)\}/(k_\text{B}T)} {\mathrm e}^{-E_j/(k_\text{B}T)},
\end{equation}
习惯上,我们将到处乱跑的因子$1/k_\text{B}T$记做
\begin{equation}
\beta \equiv 1/(k_\text{B}T)
\end{equation}
另外,$U-TS(U)$是系统的Helmholtz势,因此最后可以得到子系统处于状态$f_j$的概率为
\begin{equation}
f_j = {\mathrm e}^{\beta F}{\mathrm e}^{-\beta E_j}.
\label{equ16.7}
\end{equation}

当然,Helmholtz势具体是多少,我们是不知道的,而是得把它给算出来。计算的关键在于,注意到\eqref{equ16.7}式中${\mathrm e}^{\beta F}$与态无关,而仅仅是扮演一个归一化因子的角色
\begin{equation}
\sum\limits_j f_j = {\mathrm e}^{\beta F}\sum\limits_j {\mathbf e}^{-\beta E_j} = 1,
\end{equation}
或者写作
\begin{equation}
{\mathrm e}^{-\beta F}=Z,
\label{equ16.9}
\end{equation}
其中“正则配分求和”$Z$定义为%
\mpar{译注:这个量现在常被称为正则配分函数。}
\begin{equation}
Z\equiv \sum\limits_j {\mathrm e}^{-\beta E_j}.
\label{equ16.10}
\end{equation}

{\it 至此,我们得到了一套计算正则系综的完整算法。给出一个系统的全体可行态$j$及其能量$E_j$,计算配分求和\ref{equ16.10}式,由此可以得到配分求和作为温度(或$\beta$)以及影响能级的其他外参量($V,N_1,N_2,\dots$)的函数。由\eqref{equ16.9}式可以得到Helmholtz势作为$T,V,N_1,N_r$的函数。这就是我们所需要的基本关系。}

整个算法归结为
\begin{equation}
-\beta F = \ln \sum\limits_j \mathrm e^{-\beta E_j} \equiv \ln Z,
\end{equation}
读者可得好好记着这个式子。

注意到$f_j$是占据状态$j$的概率,\eqref{equ16.7}、\eqref{equ16.9}及\eqref{equ16.10}式可以改写如下形式
\begin{equation}
f_j  = {\mathrm e}^{-\beta E_j}/\sum\limits_i \mathrm e^{-\beta E_i},
\end{equation}
平均能量自然就是
\begin{equation}
U = \sum\limits_j E_jf_j = \sum\limits_j E_j{\mathrm e}^{-\beta E_j}/\sum\limits_i \mathrm e^{-\beta E_i},
\label{equ16.12}
\end{equation}
或者写成
\begin{equation}
U = -(\mathrm d/\mathrm d\beta)\ln Z
\label{equ16.13}
\end{equation}
带入\eqref{equ16.9}式,用$F$表示$Z$,并记住$\beta = 1/k_\text{B}T$,可以验证热力学中早已得到的一个关系式$U=F+TS=F-T(\partial F/\partial T)$。\eqref{equ16.12}和\eqref{equ16.13}式在统计力学中非常有用,但得强调一遍它们并不算是基本关系。基本关系由\eqref{equ16.9}和\eqref{equ16.10}式给出,为$F$(而不是$U$)作为$\beta,V,N$的函数。

对单位和整体结构作一个回顾将是有启发性的。$\beta$,作为倒数温度,是一个“自然单位”。正则系综用$\beta,V$和$N$表示出了$\beta F$。即,$F/T$作为$1/T,V$以及$N$的函数被给出。{\it 这就是在$S[1/T]$表象下的基本方程}(请回忆\ref{sec5.4}节)。正如同微正则系综理论自然的给出熵表象一样,正则系综理论自然的给出$S[1/T]$表象。而我们在\ref{chap17}中所要讨论的巨正则系综理论,将自然给出Massieu函数。当然我们还是得记住正则系综是基于Helmholtz势的,可别弄错了表象%
\mpar{译注:原文为``misrepresentation'',为“歪曲、误传”之意,此处应有双关``mis-representation'',弄错表象之意}%
。

{\noindent\bf 习题}
\begin{itemize}
\item[16.1-1] 证明\eqref{equ16.13}式等价于$U=F+TS$。
\item[16.1-2] 从\eqref{equ16.9}和\eqref{equ16.10}式给出的正则算法出发,将压强用配分求和的某个导数表示出来。进一步的,将压强用导数$\partial E_j/\partial V$(以及$T$和$E_j$)表示出来。你能否对这个式子给一个有启发性的解释?
\item[16.1-3] 证明$S/k_\text{B}=\beta^2\partial F/\partial\beta$,并将$S$用$Z$与其对$\beta$的导数表示。
\item[16.1-4] 证明$c_v=-\beta(\partial s/\partial\beta)_v$,将$c_v$用配分求和及其对$\beta$的导数表示。
\begin{flushleft}
{\it 答案}\\
$c_v=N^{-1}k_\text{B}\beta^2\frac{\partial^2\ln Z}{\partial \beta^2}$
\end{flushleft}
\end{itemize}


\section{可加的能量与配分求和的可分性}\label{sec16.2}

为了展现正则系综理论在实际问题上显著的简易性,我们重新回顾一下\ref{sec15.3}节中所提到的二态模型。$\tilde N$个可分辨的``原子'',各自可能处于两个态上,其能量分别为$0$和$\varepsilon$。对于微正则系综理论而言,即便是将问题拓展到仅仅三个态上,单去求激发能量,这也是难以求解的。而正则系综理论则能相当简单的处理这个问题!

考虑一个系统由$\tilde N$个可分辨的``单元''构成,每个单元都对应于系统一个独立(无相互作用的)的激发模式。对于由无相互作用的物质成分组成的系统,如理想气体分子,这里``单元''就对应于单个的分子。而对于强相互作用的系统,单元可能对应于某种波状的集体激发,例如振动模式或者电磁模式。{\it ``单元''的标识性特征是,系统的总能量等于所有单元的能量之和,这些能量都是相互独立而没有相互作用的。}

每个单元可以布居在一系列{\it 轨道态(orbital states)}(此后我们使用{\it 轨道态}来描述单元的态,以与集体系统的态相区分)上。第$i$个单元处在第$j$个轨道态上的能量记为$\varepsilon_{ij}$。这些单元的能量和轨道态的数目不一定得要是一样的。{\it 系统的总能量等于各个单元能量之和,每个单元所能处在的轨道态与其他单元的布居无关。}故配分求和为
\begin{align}
Z &= \sum\limits_{j,j',j'',\dots}\mathrm e^{-\beta (\varepsilon_{1j}+\varepsilon_{2j}+\varepsilon_{3j}+\dots)}\label{equ16.14} \\
 &= \sum\limits_{j,j',j'',\dots}\mathrm e^{-\beta\varepsilon_{1j}}\mathrm e^{-\beta\varepsilon_{2j}}\mathrm e^{-\beta\varepsilon_{3j}}\dots \\
 &= \sum\limits_{j}\mathrm e^{-\beta\varepsilon_{1j}}\sum\limits_{j'}\mathrm e^{-\beta\varepsilon_{2j}}\sum\limits_{j''}\mathrm e^{-\beta\varepsilon_{3j}}\dots \\
 &= z_1z_2z_3\dots, \label{equ16.17}
\end{align}
其中$z_i$为“第$i$个单元的配分求和”,定义为
\begin{equation}
z_i = \sum\limits_{j}\mathrm e^{-\beta\varepsilon_{ij}},
\end{equation}
{\it 配分求和因子。进一步的,Helmholtz势对于诸单元是可加的}
\begin{equation}
-\beta F = \ln Z = \ln z_1+\ln z_2+\dots, \label{equ16.21}
\end{equation}
这个结果贼简单、贼强并且贼有用,让我们不得不再强调一遍,它对于任何满足如下条件的系统适用:(a)能量等于诸单元能量之和,以及(b)每个单元所能处在的轨道态与其他单元的布居无关。

\ref{sec15.3}节中提到的“二态模型”满足上述的条件,即
\begin{equation}
Z = z^{\tilde N}= (1+\mathrm e^{-\beta\varepsilon})^{\title N},
\end{equation}
和
\begin{equation}
F = -\tilde Nk_\text{B}T\ln(1+\mathrm e^{-\beta\varepsilon}).
\end{equation}
证明其与\ref{sec15.3}节的结果的等价性的任务留给读者。如果轨道数是三而不是二,单粒子配分求和$z$就有三项,而Helmholtz势对数函数的变量中也会多加一项。

Einstein晶体模型(\ref{sec15.2}节)也能体现正则系综理论的简洁性。这里“单元”是振动模式,单个模式的配分求和为
\begin{equation}
z = 1 + \mathrm e^{-\beta\hbar\omega_0} + \mathrm e^{-2\beta\hbar\omega_0} + \mathrm e^{-3\beta\hbar\omega_0} + \dots = \sum\limits_{n=0}^{\infty} \mathrm e^{-n\beta\hbar\omega_0},
\end{equation}
“几何级数”\mpar{译注:即等比级数}的和为
\begin{equation}
z = \frac{1}{1-\mathrm e^{-\beta\hbar\omega_0}}.
\end{equation}
由于存在着$3\tilde N$个振动模式,故在正则系综理论下,Einstein模型的基本方程为
\begin{equation}
F = -\beta\ln z^{3\tilde N}=3\tilde Nk_\text{B}T\ln(1-\mathrm e^{-\beta\hbar\omega_0}).
\label{equ16.24}
\end{equation}
显然,在这套框架下,Einstein关于所有振动模式的频率都相同的简化假设成了没有必要的。在\ref{sec16.7}节中,我们将讨论由P. Debye提出的一个物理上更可信的假设。

\noindent{\bf 习题}
\begin{itemize}
\item[16.2-1] 考虑一个包括三个不同粒子的系统。第一个粒子有两个轨道态,能量分别为$\varepsilon_{11}$和$\varepsilon_{12}$。第二个粒子可能的能量分别为$\varepsilon_{21}$和$\varepsilon_{22}$,第三个粒子是$\varepsilon_{31}$和$\varepsilon_{32}$。根据\eqref{equ16.14}式写下配分求和,并通过详细的过程将其化成\eqref{equ16.17}式的形式。
\item[16.2-2] 证明,对于二能级系统,通过\eqref{equ16.21}式算出的Helmholtz势与\ref{sec15.3}节中给出的基本方程等价。 
\item[16.2-3] 考虑无电荷质点作为气体粒子(并略去引力相互作用),其能量是否是可加的?若半数粒子带正电,另外半数粒子带负电,配分求和是否是可分的?如果粒子是遵循Pauli不相容原理的fermion(例如中微子),配分求和是否是可分的?
\item[16.2-4] 根据\eqref{equ16.24}式计算单个模式的热容。
\item[16.2-5] 根据\eqref{equ16.24}式计算单个模式的能量。当$T\rightarrow 0$以及$T\rightarrow \infty$时,$U(T)$的领头项是什么?
\item[16.2-6] 某种二元合金由$\tilde N_A$个$A$类原子和$\title N_B$个$B$类原子构成。每个$A$类原子分别可以处在基态和一个激发态上,其间有能量差$\varepsilon$(其他态的能量都太高了,以至于在所考虑的温度范围内没有影响)。每个$B$类原子同样分别可以处在基态和一个激发态上,其间有能量差$2\varepsilon$。整个系统处于温度$T$。
	\begin{itemize}
	\item[a)] 计算系统的Helmholtz势。
	\item[b)] 计算系统的热容。
	\end{itemize}
\item[16.2-7] 某类顺磁盐由\SI{1}{\mole}无相互作用的离子构成,各自拥有一单位Bohr磁矩($\mu_\text{B}=\SI{9.274e-24}{\joule\per\tesla}$)。磁场$B_e$给定在某方向上,离子可能分别处在磁矩方向平行或反平行于磁场方向的态上。
	\begin{itemize}
	\item[a)] 假定系统温度维持在$T=\SI{4}{\kelvin}$,$B_e$从\SI{1}{\tesla}增大至\SI{10}{\tesla},热库中流出了多少热量?
	\item[b)] 假定系统与外界绝热,$B_e$从\SI{10}{\tesla}减小至\SI{1}{\tesla},系统末态的温度是多少?(这个过程被称为绝热去磁降温。)
	\end{itemize}
\end{itemize}

\section{气体的内部模式}\label{sec16.3}

对于分子气体,其激发包括三个整体平动模式、振动模式、转动模式、电子的模式以及原子核的激发模式。为简单起见,我们暂且先假定这些模式之间是独立的,后面再来检验这个假定的合理性。配分求和对于这些模式是可分的
\begin{equation}
Z = Z_\text{平动}Z_\text{振动}Z_\text{转动}Z_{电子}Z_{核},
\end{equation}
进一步的,可以分解成对于单个分子的乘积
\begin{equation}
Z_\text{振动}=z_\text{振动}^{\tilde N},\quad Z_\text{转动}=z_\text{转动}^{\tilde N},
\end{equation}
对于电子和核的配分求和也是类似的。

气体是否是“理想的”将会影响到平动配分求和。其需要特别的小心处理,我们将这一部分内容推后放在了\ref{sec16.10}节。现在先简单的假定任何分子间的碰撞都不会与内部模式(转动、振动等)相耦合。

$\tilde N$个同一类振动模式(分别分布在各个分子上)与Einstein 晶体模型原则上都是一样的,即谐振子。对于频率为$\omega_0$的成分
\begin{equation}
Z_\text{振动}=z_\text{振动}^{\tilde N}=(1-\mathrm e^{-\beta\hbar\omega_0})^{-\tilde N},
\end{equation}
该振动模式对Helmholtz势的贡献由\eqref{equ16.24}式给出(将$3\tilde N$替换成$\tilde N$)。其对热容的贡献在图\ref{fig15.2}中给出(此时纵坐标单位应该是$c/R$而不是$c/3R$)。正如\ref{sec13.1}节中所描述的一样,热容在$k_\text{B}T\simeq \hbar\omega_0$附近“呈阶梯状地上升”了一个渐进于$c=R$的量。图\ref{fig13.1}所绘制的包括了两个振动模式的贡献,满足$\omega_2=15\omega_1$。

典型的振动温度$\hbar\omega_0/k_\text{B}$从数千开尔文(对于包含较轻部件的分子,例如\ce{H_2}约为\SI{6300}{\kelvin}),到数百开尔文(对于包含较重部件的分子,例如\ce{Br_2}是\SI{309}{\kelvin})。

我们从异核双原子分子(例如\ce{HCl})出发讨论气体中的转动模式;为了描述其取向,我们需要两个角度坐标。这类异核双原子分子的转动能量是量子化的,其能量本征值由下式给出
\begin{equation}
\varepsilon_\ell = \ell(\ell+1),\quad \ell =0,1,2,\dots
\label{equ16.28}
\end{equation}
诸能级的简并度为$(2\ell+1)$。能量单位$\varepsilon$等于$\frac{1}{2}\hbar^2/(\text{转动惯量})^2$,对于\ce{HCl}来说约等于\SI{2e-21}{\joule}。典型的能级间隔和$\varepsilon$的量级相当,\ce{HCl}这里大概对应于温度$\varepsilon/k_\text{B}\simeq \SI{15}{\kelvin}$——当然,比轻分子更大,比重分子更小。

单个分子的转动配分求和是
\begin{equation}
z_\text{转动}=\sum\limits_{\ell=0}^{\infty}(2\ell+1)\mathrm e^{-\beta\ell(\ell+1)\varepsilon}.
\label{equ16.29}
\end{equation}
如果有$k_\text{B}T\gg\varepsilon$,求和可以由积分近似给出。注意到$2\ell+1$正好是$\ell(\ell+1)$的导数,故可将积分变量$x$选作$\ell(\ell+1)$
\begin{equation}
z_\text{转动}\simeq\int_{0}^{\infty}\mathrm e^{-\beta\varepsilon x}\,\mathrm dx =\frac{1}{\beta\varepsilon}=\frac{k_\text{B}T}{\varepsilon}.
\end{equation}

如果$k_\text{B}T$比$\varepsilon$小,亦或是同一个量级,那么实践中我们通常精确计算级数的前$\ell'$项,满足$\ell'(\ell'+1)\gg k_\text{B}T$,然后通过积分(从$\ell'$积到无穷大)近似计算后面的那些项。具体可以参考习题16.3-2。

读者可以尝试证明,当$k_\text{B}T\gg\varepsilon$时,平均能量为$k_\text{B}T$。

对于同核双原子分子,例如\ce{O_2}或者\ce{H_2},其受到一些量子的对称性条件的约束,我们不会在这里做具体讨论。对于不同的同核双原子分子,配分求和中仅有偶数项或奇数项才是允许的。在高温极限下,这个约束的影响等价于给单分子配分函数简单的除二。

原子核以及电子的贡献也可以用类似的办法来算,但一般来说只有它们各自的基态才会有贡献。$z_\text{核}$简单的等于其基态的简并度。这些因子在Helmholtz势中仅贡献一项$\tilde Nk_\text{B}T\ln(\text{简并度})$。

重新讨论最开始对于这些模式独立性的假定会是有意思的。这个假定一般来说是一个好的(但{\it 并非}是严格的)近似。双原子分子的振动会改变核间距,从而影响转动惯量。这只是因为一般来说振动相对转动来说相对很快,振动模式只能感受到一个{\it 平均}的核间距,从而相当于与振动态相独立。

\noindent{\bf 习题}
\begin{itemize}
\item[16.3-1] 在$k_\text{B}T\gg \varepsilon$的区域中计算异核双原子分子的单分子平均转动能量以及其对热容的贡献。
\item[16.3-2] 在\eqref{equ16.29}式中,通过精确计算前两项并用积分近似替代剩下的项,来计算转动对于单分子Helmholtz势的贡献。Euler-McLaurin求和公式会对这个问题有帮助
\begin{equation*}
\sum\limits_{j=0}^{\infty}f(j) \simeq \int_0^{\infty}f(\theta)\,\mathrm d\theta +\frac{1}{2}f(0)-\frac{1}{12}f'(0)+\dots
\end{equation*}
式中$f'(0)$代表$f(\theta)$的导数。
\item[16.3-3] 某种同核双原子分子气体具有一个振动模式,频率为$\omega$,转动能量参数为$\varepsilon$(\eqref{equ16.28}式)。假定分子之间没有相互作用,即气体为理想气体。完整计算在温度满足$T\gg \varepsilon/k_\text{B}$以及$T\simeq \hbar\omega/k_\text{B}$时体系的基本方程。
\end{itemize}

\section{可分系统中的概率}\label{sec16.4}

\section{小系统的统计力学:系综}\label{sec16.5}

\section{态密度与轨道态密度}\label{sec16.6}

\section{非金属晶体的Debye模型}\label{sec16.7}

\section{电磁辐射}\label{sec16.8}

\section{经典态密度}\label{sec16.9}

\section{经典理想气体}\label{sec16.10}

\section{高温下的性质——能均分定理}\label{sec16.11}