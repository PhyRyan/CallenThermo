%!TEX root = ../CallenThermo.tex
\chapter{正则系综:Helmholtz 表象下的统计力学}\label{chap16}
\section{概率分布}\label{sec16.1}
上一章所讲的微正则系综在原理上看起来很简单,但实际计算起来却仅仅对少数几种高度理想化的系统适用。计算在任意大小的“盒子”里分配给定能量的方案数量往往超出了我们的数学能力。而将能量给定这个约束去除的方案是存在的——考虑与热库所接触的系统,而不是绝热的。与热库相接触的系统的统计力学可以看做是在“Helmholtz表象”下的,或者用这儿的术语,在{\it 正则系综~(canonical formalism)}下的。

对于一个与热库接触的系统而言,从零到任意高能量的全部态都是可能的。但是,与封闭系统相比,这里在各个态上并{\it 没}有相同的概率。即,系统在各个态上停留的时间并不相同。正则系综的关键在于确定系统处在诸微观态上的概率。这可以通过考察这个系统与热库共同组成的{\it 封闭}系统来得到解决,对于这个大系统,微观态的等概率原理依旧适用。

我们可以通过一个简单的例子来说明。考虑三个骰子,其中一个是红的,另两个白的。三个骰子都掷了数千次。记录当且仅当三个骰子数之和为$12$时,红色骰子的点数。那么红色骰子点数为一、二、\dots 、六的频率各是多少呢?

结果留给读者:红色骰子点数为一的频率是$2/25$、为二的概率为$3/25$、\dots 、为五的概率为$6/25$,为六的概率为$5/25$。在这个约束下,掷出一个六点的红骰子概率是$1/5$。

这个红色骰子就正如我们所关心的系统一样,而白色骰子好比热源,点数对应于能量,总点数为$12$的限制又与(系统与热库的)总能量为常数类似。

{\it 子系统处于宏观状$j$的概率$f_j$,等于子系统处于宏观状$j$(其能量为$E_j$)的微观状态数比上系统和热库全部微观状态数:}
\begin{equation}
f_j = \frac{\Omega_\text{热库}(E_\text{tot}-E_j)}{\Omega_\text{tot}(E_\text{tot})},
\end{equation}
这里$E_\text{tot}$为系统与热库的能量之和,$\Omega_\text{tot}$为系统与热库的全部状态数。分子上的$\Omega_\text{热库}(E_\text{tot}-E_j)$为子系统处于状态$j$时(给热库留下了$E_\text{tot}-E_j$的能量)热库的全部可能状态数。

这是正则系综理论里最精华%
\footnote{译注:原文为`seminal',意为含精液的,引申为有重大意义的。}%
的一个关系式,其可以重新改写成一个更便利的形式。分母项可以通过\eqref{equ15.1}式用复合系统的熵来表达,而分子和热库的熵相关。这样我们有
\begin{equation}
f_j = \frac{\exp\left\{k_\text{B}^{-1}S_\text{热库}(E_\text{tot}-E_j)\right\}}{\exp\left\{k_\text{B}^{-1}S_\text{tot}(E_\text{tot})\right\}} ,
\end{equation}
记$U$为子系统的平均能量,从熵的可加性,我们有
\begin{equation}
S_\text{tot}(E_\text{tot}) = S(U) + S_\text{热库}(E_\text{tot}-U).
\end{equation}
此外,将熵$S_\text{热库}(E_\text{tot}-E_j)$在平衡点$E_\text{tot}-U$附近展开
\begin{equation}
\begin{aligned}
S_\text{热库}(E_\text{tot}-E_j)& = S_\text{热库}(E_\text{tot}-U+U-E_j)\\
&= S_\text{热库}(E_\text{tot}-U) + (U-E_j)/T
\end{aligned}
\end{equation}
展开式中不存在更多的项(这也是热库的定义)。将后两个方程带入$f_j$的表达式
\begin{equation}
f_j = {\mathrm e}^{\{U-TS(U)\}/(k_\text{B}T)} {\mathrm e}^{-E_j/(k_\text{B}T)},
\end{equation}
习惯上,我们将到处乱跑的因子$1/k_\text{B}T$记做
\begin{equation}
\beta \equiv 1/(k_\text{B}T)
\end{equation}
另外,$U-TS(U)$是系统的Helmholtz势,因此最后可以得到子系统处于状态$f_j$的概率为
\begin{equation}
f_j = {\mathrm e}^{\beta F}{\mathrm e}^{-\beta E_j}.
\label{equ16.7}
\end{equation}

当然,Helmholtz势具体是多少,我们是不知道的,而是得把它给算出来。计算的关键在于,注意到\eqref{equ16.7}式中${\mathrm e}^{\beta F}$与态无关,而仅仅是扮演一个归一化因子的角色
\begin{equation}
\sum\limits_j f_j = {\mathrm e}^{\beta F}\sum\limits_j {\mathbf e}^{-\beta E_j} = 1,
\end{equation}
或者写作
\begin{equation}
{\mathrm e}^{-\beta F}=Z,
\label{equ16.9}
\end{equation}
其中“正则配分求和”$Z$定义为%
\footnote{译注:这个量现在常被称为正则配分函数。}
\begin{equation}
Z\equiv \sum\limits_j {\mathrm e}^{-\beta E_j}.
\label{equ16.10}
\end{equation}

{\it 至此,我们得到了一套计算正则系综的完整算法。给出一个系统的全体可行态$j$及其能量$E_j$,计算配分求和\ref{equ16.10}式,由此可以得到配分求和作为温度(或$\beta$)以及影响能级的其他外参量($V,N_1,N_2,\dots$)的函数。由\eqref{equ16.9}式可以得到Helmholtz势作为$T,V,N_1,N_r$的函数。这就是我们所需要的基本关系。}

整个算法归结为
\begin{equation}
-\beta F = \ln \sum\limits_j \mathrm e^{-\beta E_j} \equiv \ln Z,
\end{equation}
读者可得好好记着这个式子。

注意到$f_j$是占据状态$j$的概率,\eqref{equ16.7}、\eqref{equ16.9}及\eqref{equ16.10}式可以改写如下形式
\begin{equation}
f_j  = {\mathrm e}^{-\beta E_j}/\sum\limits_i \mathrm e^{-\beta E_i},
\end{equation}
平均能量自然就是
\begin{equation}
U = \sum\limits_j E_jf_j = \sum\limits_j E_j{\mathrm e}^{-\beta E_j}/\sum\limits_i \mathrm e^{-\beta E_i},
\label{equ16.12}
\end{equation}
或者写成
\begin{equation}
U = -(\mathrm d/\mathrm d\beta)\ln Z
\label{equ16.13}
\end{equation}
带入\eqref{equ16.9}式,用$F$表示$Z$,并记住$\beta = 1/k_\text{B}T$,可以验证热力学中早已得到的一个关系式$U=F+TS=F-T(\partial F/\partial T)$。\eqref{equ16.12}和\eqref{equ16.13}式在统计力学中非常有用,但得强调一遍它们并不算是基本关系。基本关系由\eqref{equ16.9}和\eqref{equ16.10}式给出,为$F$(而不是$U$)作为$\beta,V,N$的函数。

对单位和整体结构作一个回顾将是有启发性的。$\beta$,作为倒数温度,是一个“自然单位”。正则系综用$\beta,V$和$N$表示出了$\beta F$。即,$F/T$作为$1/T,V$以及$N$的函数被给出。{\it 这就是在$S[1/T]$表象下的基本方程}(请回忆\ref{sec5.4}节)。正如同微正则系综理论自然的给出熵表象一样,正则系综理论自然的给出$S[1/T]$表象。而我们在\ref{chap17}中所要讨论的巨正则系综理论,将自然给出Massieu函数。当然我们还是得记住正则系综是基于Helmholtz势的,可别弄错了表象%
\footnote{译注:原文为``misrepresentation'',为“歪曲、误传”之意,此处应有双关``mis-representation'',弄错表象之意}%
。

{\noindent\bf 习题}
\begin{itemize}
\item[16.1-1] 证明\eqref{equ16.13}式等价于$U=F+TS$。
\item[16.1-2] 从\eqref{equ16.9}和\eqref{equ16.10}式给出的正则算法出发,将压强用配分求和的某个导数表示出来。进一步的,将压强用导数$\partial E_j/\partial V$(以及$T$和$E_j$)表示出来。你能否对这个式子给一个有启发性的解释?
\item[16.1-3] 证明$S/k_\text{B}=\beta^2\partial F/\partial\beta$,并将$S$用$Z$与其对$\beta$的导数表示。
\item[16.1-4] 证明$c_v=-\beta(\partial s/\partial\beta)_v$,将$c_v$用配分求和及其对$\beta$的导数表示。
\begin{flushleft}
{\it 答案}\\
$c_v=N^{-1}k_\text{B}\beta^2\frac{\partial^2\ln Z}{\partial \beta^2}$
\end{flushleft}
\end{itemize}
