\chapter{一级相变}
\label{chap9}

\section{能量最小原理}
\label{sec9.1}

通常情况下在室温和大气压的条件下水是液体,
但是如果被冷却到$273.15K$以下就会凝固;
而如果被加热到$373.15K$以上就会汽化。
在这些温度下物质的性质都会发生急剧的变化——也就是“相变”。
在高压下水会经历一些额外的从一种固态到另一种固态的相变过程。
这些不同的固相,被定为“冰I”,“冰II”,“冰III”,……,在晶体结构以及基本所有热力学性质上都不同(比如压缩系数,摩尔热容,以及各种摩尔势比如$u$或者$f$)。
水的“相图”如图\ref{fig9.1}所示。

每一个相变都对应着热力学基本关系的一个线性区域(比如图\ref{fig8.2}的$BHF$),
并且每一个都可以看成潜在的基本关系中稳定条件(凸的或者凹的)失效的结果。

在本节中我们要考虑基本关系不稳定的系统。
通过定性考虑这种系统的涨落我们会看到{\it 涨落会受到潜在的基本关系的细节的深刻影响}。
与此相对的,{\it 广延量的均值只能反应稳定的热力学基本关系。}

关于潜在的基本关系的形式对于热力学涨落的影响方式的考虑会给
第\ref{chap8}章中稳定性的考虑以及图\ref{fig8.2}的构造(其中热力学基本关系是通过切平面的包络构造出来的)提供物理解释。

一个简单的力学模型可以通过一个直观易懂的类比来解释这种考虑。
考虑一段半圆形的两端封闭的管子。
这个管子如同倒着的$U$型垂直竖立在桌子上(图\ref{fig9.2})。
这个管子内包含一个可以自由移动的活塞将其分成两部分,
每一部分都包含一摩尔的气体。
这个系统的对称性会被证明有重要的效果,
而为了破坏这个对称性我们考虑管子的每一段含有一个小的金属“滚珠”(也就是一个小的金属球)。
两个金属球是热膨胀系数不同的两种金属构成的。

在一些特别的温度下,比如我们记为$T_c$,两个金属球的半径相同;
而在温度高于$T_c$的时候,右边的球更大。

活塞暂时被放在管子的顶端,可以落入两条腿中的任意一个,
从而压缩那一条腿中的气体,而另一条腿中的气体则会膨胀。
在这两种互斥的平衡态中,压强差都正好被活塞质量的效果补偿。

如果没有这两个滚珠的话这两个互斥的平衡态是完全等价的。
但当存在滚珠的时候,如果$T>T_c$那么落入左边是一个更加稳定的平衡态,
如果$T<T_c$那么落入右边就成了更加稳定的平衡态。

从热力学的观点来看,系统的 Helmholtz 势是$F=U-TS$,
而能量$U$包含了活塞的重力势能以及我们所熟悉的两部分气体的热力学能量
(当然也包括两个滚珠的内能,不过我们假定这部分很小而且两者相同)。
从而系统的Helmholtz势就有两个局域极小值,
而更低的哪一个极小值对应着活塞落入包含的球更下的那一侧。

活塞从一边到另一边的平衡位置之间的移动也可以通过在给定温度下倾斜桌子达到类似的效果——或者在类似的热力学中,通过加入一些温度以外的热力学参数实现。

平衡态从一个局域极小移动到另一个构成了{\it 一级相变},
这是由于温度或者其它热力学参数的改变导致的。

{\it 一级相变联系的两个态是有区别的,它们出现在热力学构型空间中独立的区域。}

为了讨论“临界现象”和“二级相变”(第\ref{chap10}章),
暂时考虑滚珠全同或者不存在的情况会很有用。
于是在低温时两个互斥的极小值是等价的。
不过随着温度的升高,活塞的两个平衡位置会在管子中上升接近顶端。
超过一个特别的温度$T_{cr}$以后,就只有活塞位于管子的顶端这一个平衡位置了。
相反地,如果把温度从$T>T_{cr}$降到$T<T_{cr}$,
这唯一的平衡态会分成两个(对称的)平衡态。
这个温度$T_{cr}$就是“临界温度”而在$T_{cr}$的变化就是“二级相变”。

{\it 二级相变联系的两个态在热力学构型空间中是连续的。}

在这一章中我们考虑一级相变。
二级相变会在第\ref{chap10}章中讨论。
在那里我们也会对这个“力学模型”做定量地考虑,尽管在这里我们仅仅做了定性的讨论。

回到两个球不同的情况,考虑活塞位于更大的极小值——
也就是跟更大的滚珠位于管子的同一侧。
处于Helmholtz势那样的一个极小值时,
即使经历热涨落(“布朗运动”)活塞也会暂时保持在那个极小值处。
在足够场的时间后一个巨大的涨落会带动活塞“越过顶端”并达到稳定的极小值。
之后它就会呆在这个更深的极小值处知道一个更大(也更加不可能)的扰动把它待会次稳定的极小值处,然后这整个过程会反复重现。
随着幅度的增大涨落的概率会下降得非常快(我们会在第\ref{chap19}章看到这一点)
以至于{\it 在大部分时间中系统会位于更加稳定的极小值处。}
宏观热力学中多有这些动力学都被忽略了,因为它只关心稳定的平衡态。

为了在更加偏向热力学的环境中讨论相变的动力学,
把我们的注意力转移到更加熟悉的热力学系统会更加方便,
它的热力学势同样具有两个局域极小值并且被一段凹的不稳定区域分开。
特别地我们考虑一个容器中压力为一个大气压而且温度高于$373.15K$的水蒸气(也就是高于水的“常规沸点”)。
我们把注意力集中到一个小的子系统——一个半径大小(可以改变)使得任意时刻都包含有一毫克水的球形区域。
这个子系统等价于跟一个热库和一个蓄压器接触,
而平衡条件是子系统的Gibbs势$G(T,P,N)$处于极小值。
由平衡条件决定的两个独立变量是子系统的能量$U$和体积$V$。

如果Gibbs势形如图\ref{fig9.3}所示那样,
其中$X_j$是体积,那么系统在更低的极小值处是稳定的。
这个极小值对应着比第二局域极小值更大的体积(或者更小的密度)。

考虑体积涨落的行为。
这种涨落是自发而且连续发生的。
图\ref{fig9.3}中曲线的斜率表征了一个强度量(在当前情况下与压强不同),
它扮演着符合Le Chatelier 原理使系统密度趋于均匀的回复“力”的角色。
偶尔涨落会太大以至于使得系统越过极大值,达到第二个极大值的区域。
接下来系统就会呆在第二个极小值区域——不过只会呆一小会儿。
要跨过第二个极小值处更低的势垒,一个更小(因此也就更加常见)的涨落就足够了。
系统很快就回到了它的稳定态。
因此很小的高密度液滴(液相!)偶尔会在气体中形成,存在一小会儿然后消失。

如果第二个极小值距离最小值很远,而且它们之间的势垒很高,
那么从一个到另一个的涨落就会变得非常不可能。
在第\ref{chap19}章中会说明,这类涨落的概率会随着中间自由能的势垒的高度指数下降。
在固态系统(其中相互作用能量比较高)中由于中间势垒高得让从一个极小值变到另一个的时间的量级超过了宇宙的年龄,所以通常不会有多个极小值。
陷入这种“亚稳态”极小值的系统{\it 实际上}处于稳定平衡态(就像更深的极小值并不存在一样)。

回到水蒸气处于高于“沸点”的某个温度的情形,假设我们降低整个系统的温度。
Gibbs势的变化形式如同图\ref{fig9.4}中显示的那样。
在温度$T_4$两个极小值会相等,而在这个温度以下,高密度(液体)相会变成绝对稳定的。
因此$T_4$就是相变的温度(在给定的压力下)。

如果蒸汽从相变温度缓慢冷却,系统会发现之前所处的绝对稳态现在变成了亚稳态。
迟早系统的一个涨落就会“发现”真正的稳态,形成液体凝结核。
这个核会迅速变大,然后整个系统会突然经历相变。
实际上系统通过一个“试探”涨落发现更稳定的态所需的时间在蒸汽到液体凝结的情况下短到难以观察。
不过在从液体到冰的持续时间在一个纯粹的情况下很容易观察到。
被冷却到低于凝固点(结冰)温度的液体被称为是“过冷的”。
轻轻敲一下容器,产生的纵波就会导致多个“密集”和“稀疏”的区域,
而这些外部诱导的涨落会代替自发的涨落引发剧烈的相变。

当把Gibbs势的极小处的值相对温度画出来的时候会引出一个有用的视角。
这个结果如图\ref{fig9.5}所示。
如果从图\ref{fig9.4}中选取极小值那么久会只有两个这种曲线,
不过任意数值都是可能的。
在平衡态最小的极小值是稳定的,所以真正的Gibbs势是图\ref{fig9.5}中的曲线的下包络。
熵的不连续(也就是潜热)对应着包络函数斜率的不连续。

图\ref{fig9.5}需要增加一个额外的维度,增加的坐标$P$作用类似于$T$。
于是Gibbs势就被表示成下包络曲面,和三个单个的相面重合。
这些曲面之间相交曲线在$P-T$平面上的投影就是我们熟悉的相图(比如图\ref{fig9.1})。

相变发生在系统从一个包络曲面越过相交线到另一个包络曲面时所处的态。

图\ref{fig9.4}的变量$X_j$或者$V$可以是任意广延量。
在顺磁到铁磁的相变中,$X_j$是磁矩。
在从一种晶体形式到另一种晶体形式的相变中(比如从立方体到六边形)
相应的参数$X_j$是一个晶体对称性变量。
在溶解度的相变中则可能是一种组分的摩尔数。
随后我们会看到这类相变的例子。
这些都符合前面描述的一般模式。

在一级相变中两个相的摩尔Gibbs势是相等的,但是其它摩尔势(比如$u,t,f$)
在相变前后是不连续的,摩尔体积和摩尔熵也是如此。
两种相占据了“热力学空间”的不同区域,而除了Gibbs势以外任何性质的相等都只是巧合。
摩尔势的不连续性就是定义一级相变的性质。

如图\ref{fig9.6}所示,如果沿气液共存曲线远离固相(也就是向着更高的温度),
摩尔体积和摩尔能量的间断会逐渐变小。
两种相会变得越来越相似。
最再气液共存曲线的终点,两种相会变得不可分辨。
一级相变会退化成一个更加微妙的相变,也就是我们将会在第\ref{chap10}章中介绍的{\it 二级相变}。
共存曲线的终点被称为{\it 临界点}。

临界点的存在阻碍了我们明确区分通称为{\it 气体}和通称为{\it 液体}的可能性。
在通过一级相变穿过气液共存曲线的时候,我们区分了两种态,
一个“明确”为气体而另一个“明确”为液体。
但从其中一个开始(比如液体,处于共存曲线上方)
我们可以沿着任意一条绕过临界点的路径到达另一个态(“气”态)而不经历相变!
从而通称的{\it 气体}和{\it 液体}相较于严格定义的记法更多是直觉上的含义。
液体和气体共同组成了{\it 流体相}。
除此以外我们在气液一级相变中对于“液相”和“气相”应该遵循标准的用法。

在图\ref{fig9.1}中有另一个非常有趣的点:气液共存曲线的另一个终点。
这个点是三条共存曲线的交点,它也是唯一一个气相液相和固相共存的点。
这种三相相容的点被称为“三相点”——这里是水的三相点。
被唯一确定的水的三相点的温度被定义(可以是任意的)为
Kelvin温标下的$273.16K$这一数值(在第\ref{sec2.6}节中已经介绍过)。

\section{熵的不连续——潜热}
\label{sec9.2}

类似图\ref{fig9.1}的相图被共存曲线分为不同的使得某一种相是稳定态的区域。
这些曲线上的每一点两种相都具有完全一样的摩尔Gibbs势,因此两种相可以共存。
考虑水处于图\ref{fig9.1}a中“冰”这一区域的某个温度和压强下。
为了增加冰的温度,每使温度升高一Kelvin我们都要提供大概$2.1kJ/kg$的热量
(冰的比热容)。
如果用固定的速度提供热量,温度也会以几乎固定的速度增加。
但当温度到达气液共存线上的"熔化温度",温度就不再增加了。
如果继续供热,冰就会熔化形成同样温度的水。
熔化每$kg$的冰需要$335kJ$的热量。
在系统到达共存曲线后(也就是处于熔点温度),
任意时刻容器中水的多少都决定于进入容器的热量的多少。
当最终供给足够多的热量的时候,冰会完全熔化,继续加热会再次导致温度的升高——
而此时的速率是由水的比热容决定的($\simeq 4.2kJ/kg-K$).

一摩尔固体熔化需要的热量就是{\it 融化热}(或者叫{\it 熔化潜热})。
它和气相与固相的摩尔熵的差相联系。
\begin{equation}
\label{equ9.1}
\ell_{LS}=T[s^{(L)}-s^{(S)}]
\end{equation}
其中$T$是给定压强下熔点的温度。

更一般地,任何一级相变的潜热是
\begin{equation}
\label{equ9.2}
\ell=T\Delta s
\end{equation}
其中$T$是相变发生的温度而$\Delta s$是两种相的摩尔熵的差。
或者潜热可以写成两种相摩尔焓的差
\begin{equation}
\label{equ9.3}
\ell=\Delta h
\end{equation}
这可以直接从恒等式$h=Ts+\mu$
(以及摩尔Gibbs函数$\mu$在两种相中都相等这一事实)中得出。
对于很多情况每一种相的摩尔焓都被做成了表。

如果相变发生在气相和液相之间,潜热被称为{\it 汽化热},
而如果发生在气相和固相之间,则被称为{\it 升华热}。

在一个大气压下水的气液相变(沸腾)发生在$373.15K$,
而汽化潜热是$40.7kJ/mole$($540cal/g$)。

在每一种情况下系统从低温相变为高温相都需要{\it 吸收}相应的潜热。
高温相的摩尔熵和摩尔焓都比低温相的要高。

需要注意的是诱使相变发生的手段是不相干的——潜热与之无关。
除了在确定的压强下加热冰(“水平地”穿过图\ref{fig9.1}a中的共存曲线),
也可以在确定的温度下加压(“竖直地”穿过共存曲线)。
两种情况下从热库中提取的潜热是相同的。

实用的水气液共存曲线的形式由“饱和蒸汽表”给出——
名称中的“饱和”代表着蒸汽和液相处于平衡。
(“过热蒸汽表”只编写了蒸汽相的性质。)
来自于Sonntag 和Van Wylen的表9.1给出了这种饱和蒸汽表的一个例子。
每一种相的属性$s,u,v$和$h$都按照惯例列在这些表中;
每一种相变的潜热都是两种相的摩尔焓的差,或者也可以用$T\Delta s$得到。

在热物理数据手册中对于很多种其他的物质都按照这种方式给出了类似的数据。

摩尔体积和摩尔熵与摩尔能量一样,在越过共存曲线的时候是不连续的。
对于水的共存曲线来说这特别有趣。
通常的经验中冰会在液态水中漂浮。
从而固相(冰)的摩尔体积要比液相的摩尔体积{\it 更大}——
这是$H_2O$的一个不平常的特性。
更常见的情况是固相更加致密,拥有更下的摩尔体积。
$H_2O$这个特殊性质的一个平凡的后果就是冰冻水管会胀裂。
而我们在第\ref{sec9.3}中会看到作为一个补偿性的后果就是可以滑冰。
而全部后果中最根本的,是水的这个特殊性质是地球上可以存在生命所必需的。
如果并比液态水更加致密,寒冷的冬天湖和海洋的表面冻结后会沉底;
表面新的液体由于没有了冰层的保护,会继续冻结(并且下沉)
直到所有水都被冻成固态(“下面结冻”而不是“上面结冻”)。

\section{共存曲线的斜率;Clapeyron 方程}
\label{sec9.3}



\section{不稳定的等温线和一级相变}
\label{sec9.4}


\section{一级相变的普遍属性}
\label{sec9.5}

\section{多组分系统一级相变——Gibbs 相律}
\label{sec9.6}

\section{两组分系统的相图}
\label{sec9.7}