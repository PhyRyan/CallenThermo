\chapter{Legendre变换表象中的极值原理}
\label{chap6}

\section{势能最小原理}\label{sec6.1}

我们已经看到,对于特定的问题,把基本方程用特定的一组无关参数来描述最为方便,而Legendre变换则是最有效的将这种不同的描述形式联系起来。当然,如果连极值原理不能用这些不同的基本方程的描述形式来体现的话,那么这种处理上的优势也就不存在了。因此,沃尔玛接下来考虑如何将极值原理的形式进行Legendre变换。

考虑一个复合系统与热库相连。进一步假定移除某些内在的约束,我们试图找到有关平衡态性质的一些数学条件。因此我们先回顾通过能量最小原理处理这个问题的解法。

在平衡态中,系统加上热库的总能量应该最小,即
\begin{align}\label{equ6.1}
d(U+U^r)=0
\end{align}
且
\begin{align}\label{equ6.2}
d^2(U+U^r)=d^2U>0
\end{align}
此外还有等熵条件
\begin{align}\label{equ6.3}
d(S+S^r)=0
\end{align}
而在\eqref{equ6.2}中将$d^2U^r$取为0是因为其可以写成
\[\frac{\partial^2 U^r}{\partial X_j^r\partial X_k^r}dX_j^rdX_k^r \]
的形式,而起对热库来说此项为零(系数有摩尔数的倒数变化趋势)。

我们利用复合系统的特定形式的内部约束可以给出另一个推论:假如内部的壁\mpar{``Wall''}可以移动,但是仍是不可穿透\mpar{``impermeable''}的,我们有
\begin{align}\label{equ6.4}
dN_j^{(1)}=dN_j^{(2)}=d(V^{(1)}+V^{(2)})=0 \quad\text{(对于所有的$j$来说)}
\end{align}
而如果壁是刚性但是对第$k$组分可以穿透的话
\begin{align}\label{equ6.5}
d(N_k^{(1)}+N_K^{(2)})=dN_j^{(1)}=dN_j^{(2)}=dV^{(1)}=dV^{(2)}=0 \quad\text{($j\neq k$)}
\end{align}
这些方程就可以足够决定平衡态了。

\eqref{equ6.1}中的微分$dU$包含了子系统之间的热流带来的$T^{(1)}dS^{(1)}+T^{(2)}dS^{(2)}$,和复合系统内部其他过程带来的$-P^{(1)}dV^{(1)}-P^{(2)}dV^{(2)}$和$\mu_k^{(1)}dN_k^{(1)}+\mu_k^{(2)}dN_k^{(2)}$的项。将$T^{(1)}dS^{(1)}+T^{(2)}dS^{(2)}$再在\eqref{equ6.1}中利用$dU^r=T^rdS^r$我们得到
\begin{align}\notag{}
T^{(1)}dS^{(1)}+T^{(2)}dS^{(2)}+T^rdS^r&=T^{(1)}dS^{(1)}+T^{(2)}dS^{(2)}-T^rd(S^{(1)}+S^{(2)})\\
&=0\label{equ6.6}
\end{align}
只有
\begin{align}\label{equ6.7}
T^{(1)}=T^{(2)}=T^r
\end{align}

因此最终体系达到平衡态的一个好判据就是热库维持了系统的恒定的温度。而其他的平衡态的条件则取决于这个复合系统内部约束的具体形式。

到目前为止我们只回顾了能量最小原理对复合系统(子系统加上热库)的应用。我们接下来终于可以把\eqref{equ6.1}和\eqref{equ6.2}写成另外一种表象了。我们重写\eqref{equ6.1}
\begin{align}\label{equ6.8}
d(U+U^r)=dU+T^rdS^r=0
\end{align}
或者,由\eqref{equ6.3}
\begin{align}\label{equ6.9}
dU-T^rdS=0
\end{align}
或者,考虑到$T^r$是常数,可以写成
\begin{align}\label{equ6.10}
d(U-TS)=0
\end{align}
类似的,考虑到$T^r$sahib常数,而$S$是一个与之无关的变量,\eqref{equ6.2}表明\footnote{$d^2U$表示将$U$对$dS$做二阶的展开项,而\eqref{equ6.11}中的$-T^rS$则只贡献一个线性的一阶项(见附录\ref{chapappendixA}\eqref{equA.9})}
\begin{align}\label{equ6.11}
d^2U=d^2(U-T^rS)>0
\end{align}
因此,$(U-T^rS)$在平衡态下处于极小。由于$U-T^rS$与Helmholtz自由能$U-TS$的相似性,我们要检验它更多的极值性质,以及它们与Helmholtz自由能极值之间的关系。我们看到平衡态的一个关键性质就是它的各个组份系统(各子系统)温度都为$T^r$。如果我们接受它的话,我们马上就可以将平衡态的可能限制在$T=T^r$。在这上面我们会发现$U-TS$就和$U-T^rS$一致了,因此我们可以把\eqref{equ6.10}写成
\begin{align}\label{equ6.12}
dF=d(U-TS)=0
\end{align}
其中附加条件为
\begin{align}\label{equ6.13}
T=T^r
\end{align}
这就是说,平衡态在$T=T^r$的空间里最小化了Helmholtz自由能。我们从而就得到了Helmholtz自由能表象下的平衡态条件。

{\bf Helmholtz自由能最小原理}。{\it 平衡态下,系统与热库进行热接触时,系统的各个无约束的内部参数取值满足在$T=T^r$的条件下最小化Helmholtz自由能。}

这一原理的重要性体现在\eqref{equ6.8}-\eqref{equ6.10}。系统的能量加上热库的能量是最小的。但是如果说是整个的Helmholtz自由能最小化则不是一回事,因为$dF=d(U-TS)$中的$d(-TS)$表示热库能量的变化(考虑到$T=T^r$,$-dS=dS^r$)。接下来把前面的这一系列内容扩展到其他表象就很简单了。

考虑复合系统,其所有子系统都与一个常压库通过一个没有约束的壁接触。我们假定所有的系统的内部约束被解除了。平衡态的第一个条件可以写成
\begin{align}\label{equ6.14}
d(U+U^r)=dU-P^rdV^r=dU+P^rdV=0
\end{align}
或者写成
\begin{align}\label{equ6.15}
d(U+P^rV)=0
\end{align}
考虑$P=P^r$时,我们有
\begin{align}\label{equ6.16}
dH=d(U+PV)=0
\end{align}
其中附加约束
\begin{align}\label{equ6.17}
P=P^r
\end{align}
接下来,考虑到$P^r$是常数,$V$是与之无关的变量,
\begin{align}\label{equ6.18}
d^2H=d^2(U+P^rV)=d^2U>0
\end{align}
因此极值为极小值。

{\bf 焓最小原理}。{\it 平衡态下,系统与常压库进行力学接触时,系统的各个无约束的内部参数取值满足在压强等于库的压强的条件下最小化焓。}

最后,考虑一个系统与一个热\&常压库接触。同样有
\begin{align}\label{equ6.19}
d(U+U^r)=dU-T^rdS+p^rdV=0
\end{align}
考虑$T=T^r$和$P=P^r$的附加条件,我们有
\begin{align}\label{equ6.20}
dG=d(U-TS+PV)
\end{align}
其中附加约束
\begin{align}\label{equ6.21}
T=T^r\quad P=P^r
\end{align}
于是,又有
\begin{align}\label{equ6.22}
d^2G=d^2(U-T^rS+P^rV)=d^2U>0
\end{align}
我们从而得到了Gibbs表象下的平衡态条件。

{\bf Gibbs自由能最小原理}。{\it 平衡态下,系统与热\&常压库进行热\&力学接触时,系统的各个无约束的内部参数取值满足在压强和温度等于库的压强和温度的条件下最小化Gibbs自由能。}

如果某个系统由除了体积,Mole数的其他的广延量描述,对它的分析与之前的形式完全一样且我们完全知道这种一般的结果:

{\bf 一般Legendre表象变换下的能量最小原理}。{\it 平衡态下,系统与库进行关于$P_1, P_2\cdots$这些强度量的接触时,系统的各个无约束的内部参数取值满足在$P_1, P_2\cdots=P_1^r, P_2^r, \cdots$的条件下最小化热力学势能$U[P_1, P_2, \cdots]$。}


\section{Helmholtz势}
\label{sec6.2}

对于一个与热库有热接触的复合系统,其平衡态使得等温态(与热库的温度相等)流形上的Helmholtz势最小。
事实上很多过程都在有着透热壁的刚性容器中发生,比方说环境大气可以视为一个热库,对于这些情况Helmholtz势表象是相当合适的。\mpar{原文:{In practice many processes are carried out in rigid vessels with diathermal walls, so that the ambient atmosphere acts as a thermal reservoir; for these the Helmholtz potential representation is admirably suited.}}

 Helmholtz势函数是以$T$,$V$,$N_1$,$N_2$,…为自变量的自然函数(natural function)。
$T$为常数的条件减少了这个问题中变量的数量,使得$F$成为一个只与变量$V$,$N_1$,$N_2$,…有关的函数。
这与$T$的固定在能量表象中体现出的行为形成了鲜明的对比。具体来说,在能量表象中$U$是$S$,$V$,$N_1$,$N_2$,…的函数,但附加条件$T=T^r$暗示着这些变量间的一个关系。
特别地,在对状态方程$T=T(S, V, N)$的具体形式一无所知的情况下,这个附加的限制将导致能量表象中的解析过程无从下手。

作为Helmholtz势用途的一个例证我们先考虑一个复合系统,这个复合系统由两个被一个可移动的、绝热的、不可透过的壁(例如一个刚性绝热活塞)隔开的简单系统组成。
这两个子系统每一个都与温度$T^r$的热库之间有热接触。
问题是预测两个子系统的体积$V^{(1)}$和$V^{(2)}$。
我们有
\begin{equation}
\label{equ6.23}
P^{(1)}\left(T^r, V^{(1)}, N_1^{(1)}, N_2^{(1)}, ...\right)=P^{(2)}\left(T^r, V^{(2)}, N_1^{(2)}, N_2^{(2)}, ...\right)
\end{equation}
这是一个包含两个变量$V^{(1)}$和$V^{(2)}$的方程;所有其余量都是常数。封闭条件
\begin{equation}
\label{equ6.24}
V^{(1)}+V^{(2)}=V
\end{equation}
提供了另一个需要的方程,使得$V^{(1)}$与$V^{(2)}$可以被明确解出。

在能量表象中我们亦能发现压强相等,正如在方程$6.23$中那样,但此时压强是熵、体积、摩尔数的函数。
我们将需要状态方程来把熵同温度与体积联系在一起;$6.23$与$6.24$两个同时存在的方程将变成$4$个。

尽管这种从$4$个方程到$2$个的简化也许看起来仅仅是个微小的成功,在更加复杂的情形下这种简化将带来巨大的方便。
也许关于这个概念的更加大价值是Helmholtz表象让我们将我们的思考过程没有干扰地集中在我们感兴趣的子系统上,与此同时只把热库视为一个隐含的角色。
最后,由于数学技巧上的原因(这将在第$16$章中详细讲到),在Helmholtz表象下统计力学的计算将被巨大地简化,使得难以对付的计算成为可能。

对于一个与热库有接触的系统,Helmholtz势可以被解释为一定温度下可获得的功。考虑一个与热库有热接触的系统,它与一个可逆功源(reversible work source)之间有相互作用。在一个可逆过程中给可逆功源(reversible work source)输入的功与系统减少的能量相等,并且
\begin{align}
\label{equ6.25}
dW_{RWS}&=-dU-dU'=-dU-T^rdS^r \\
\label{equ6.26}
                  &=-dU+T^rdS=-d(U-T^rS) \\
\label{equ6.27}
                  &=-dF
\end{align}
因此在一个可逆过程中一个与热库有接触的系统所释放的能量与这个系统Helmholtz势减少的量相等。
Helmholtz势经常被称作Helmholtz自由能,尽管短语“一定温度下可获得的功”更不容易被误解。

\subsection*{例子1}
一个圆筒内部包含一个活塞,活塞的每一侧都有一摩尔的单原子分子理想气体。
圆筒的壁是透热的,整个系统浸入到一个温度在$0$摄氏度的大液浴(一个热库)中。
这两个气体子系统(活塞的两边)的初始体积分别为$10$升和$1$升。
活塞现在被可逆地移动,使得其两边最后的体积分别为$6$升和$5$升。
问做了多少功?
\subsubsection*{答案}
正如在问题$5.3-1$中读者见到的那样,Helmholtz势表象中单原子分子理想气体的基本方程式为
\begin{equation}
\notag
F=NRT\left\{\frac{F_0}{N_0RT_0}-ln\left[\left(\frac{T}{T_0}\right)^{3/2}\frac{V}{V_0}\left(\frac{N}{N_0}\right)\right]\right\}
\end{equation}
当$T$与$N$为常数时这意味着
\begin{equation}
\notag
F=常数-NRTlnV
\end{equation}
Helmholtz势的变化为
\begin{equation}
\notag
\Delta F=-NRT[ln6+ln5-ln10-ln1]=-NRTln3=-2.5kJ
\end{equation}
因此$2.5kJ$的功被用于这个过程。

在这里注意到一件有趣的事情,所有的能量都来自于热库。
单原子分子理想气体的能量仅仅是$\frac{3}{2}NRT$,因此它在一定的温度下是个常数。
然而,我们从热库中吸取能量并将它作为注入到可逆功源的功的事实并不违反Carnot效率原理,因为气体子系统并不处在它们的初始态上。
除了这些子系统的能量保持恒定的事实外,它们的熵增加了。
