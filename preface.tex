%----------------------------------------------------------------------------------------
%	翻译:SI
%   校对:未校对!
%----------------------------------------------------------------------------------------


\chapter*{前言}
本书第一版完成于二十五年前,我深感荣幸地看到此书已成为热力学领域常用的参考文献,书中建立的具有探索性的新体系如今已被广泛接纳。然而第一版仍有许多值得改进之处,因而我们在之前的基础上加以延伸,推出了这一新版本。

首先,热力学在19世纪60-70年代有重要进展,特别是临界相变领域。尽管这些前沿进展远远超出了这本教材的范围,但我仍尽力描述这些问题的本质,在二阶相变的基础上介绍critical exponents与scaling function。这一讲法生动简洁,并且替代了第一版当中有关二阶相变的相对复杂的理论(许多学生认为这是第一版中最难的章节)。

其次,部分内容进行教学法方面的改良,以使得它能胜任面向物理系大三、研一,物理学家,工程师和化学家的课程。我得到了许多教师与学生在这方面的建议。第二版简化了一些说明内容,并添加了大量有详细解答的例题。课后习题的数量有所增加,部分习题给出了答案或提示。

第三,添加了统计力学原理的介绍。该部分保持了第一版的风格,强调统计力学原理潜在的简洁性与逻辑思辨的培养,而不止着眼于统计力学的众多应用。为此本书避免直接从量子力学中的非对易问题入手,所需的先修知识只包括基本的量子论(例如熟悉有限系统的分立能级),这使得高年级本科生可以学习这些内容。同时更高年级的学生不难从教材的体系中导出非对易情形的理论。

第四,长期以来我对热力学基础中的一些潜在的概念性问题感到疑惑,这让我尝试建立热力学“含义”(``meaning'')的一种诠释。本书最后一章 --- 教材的“诠释性结束语” --- 介绍一种热力学\&统计力学源自基本物理定律的{\it 对称性}(而非热力学定律的定量描述)的理论。这部分只是定性的描述性内容,凭借直觉搭建起理论框架,使读者认识到热力学的自然与基本,进而体会科学理论的优美结构。


新版本包含热力学与统计力学,但本书并未完全分割它们,也没有将二者融为一体(不然题目就是“热物理学”(thermal physics)了)。我认为这两种“极端”做法都有点跑偏。将热力学从它的统计力学基础中完全剥离出来的行为让热力学丧失其基本的物理起源。缺少统计力学视角的热力学只会停留在19世纪的宏观经验层次,而忽视了19世纪至今的科学进展。相对的,将热力学与统计力学合体为“热物理学”则会使热力学比重过低。宏大深邃的统计力学(与热力学相比)理论气息过浓\mpar{原文为The fundamentality and profundity of statistical mechanics are treacherously seductive, 这里译者斗胆进行自行理解的意译。}。“热物理学”课程往往忽视了宏观运行原理。\footnote{美国物理学会委员会(The American Physical Society Committee)在 Applications of Physics [{\it Bulletin of the APS}, Vol 22 \#10, 1233 (1971)] 工业研究领导人中的调查显示在所有本科物理系科目中热力学比起其他课程需要加深学习,然而热力学的重视程度近来不断{\it 降低}。} 此外,热力学与统计力学的融合与“物理理论的经济原则”相违背:理论预言应该从尽可能一般、尽可能少的假设中导出。热力学的基本理论完全可以避开统计力学特有的假设。把“热力学”当做从属的学科组织方式是与这种思想不相容的\mpar{这句话的翻译来自超理论坛@monad,译者在此致谢。}。

本书通过如下方式平衡热学的两大部分。首先在宏观层面上介绍公式化的热力学,由此热力学基本假设与统计力学定理可以精确而清楚地表述,同时这两大部分之间的联系也会经常提及。不过教师在授课时可以根据下文的顺序表在热力学章节中交错讲授统计力学部分,即便如此,热力学的宏观结构总是在统计因素{\it 之前}引入的。这种分开、排序的方式保留与强调了科学的分层结构,将物理学按连贯的单元组织起来,并且各单元间的联系清晰而好记。与此相似的是经典力学,只有将它看成是量子力学的极限之后它才可以作为自洽的理论体系而深入理解。

下文的“菜单”是两种主要的授课顺序。一种选项是平推(菜单的A列上完再上B列)。另一种“交错”的选项是按菜单自上而下的顺序进行。第15章是简洁而基本的熵的统计诠释的内容,它可以在学完第1, 4或7章之后立即讲述。

第一条横线下面的章节是可选内容,为了平衡课程中的具体与抽象成分,教师可以讲述第13章(材料性能)的部分内容,或者第21章(对称性与概念基础)中的理论假设。

大三一学期课程(最少内容)包括本书前七章,时间充裕的话第15、16章也可加入。

\begin{flushright}
{\it Philadelphia, Pennsylvania} 

{\bf Herbert B. Callen}
\end{flushright}


\section*{第四次重印前言}
第二版第四次重印承蒙出版社厚意得以更正一些印刷错误与“微小”误差。我深切认识到任何数字或文本的错误对读者的影响都不会是“微小”的。因此我深深感谢指出错误的大量读者,以及允许更正错误的靠谱的出版社。

\begin{flushright}
1987年11月

{\it Herbert B. Callen}
\end{flushright}



\newpage
{{

\centering
\begin{tabular}{ll}
1. 理论假设 & \\
 & 15 \\
2. 平衡态条件 & \\
3. 形式关系与样例系统 & \\
4. 可逆循环,引擎 & \\
 & 15. 熵表象中的统计力学 \\
5. Legendre变换 & \\
6. Legendre变换表象的极值条件 & \\
7. Maxwell关系 & \\
 & 15 \\
 & 16. 正则形式 \\
 & 17. 广义正则形式 \\
8. 稳定性 & \\
9. 一阶相变 & \\
\hline
10. 临界现象 & 18. 量子流体 \\
11. Nernst假设 & 19. 涨落 \\
12. 原理总结 & 20. 变分性质与平均场理论 \\
13. 材料性能 & \\
14. 不可逆热力学 & \\
\hline
21. 假设:对称性与热力学概念基础
\end{tabular}
}}
